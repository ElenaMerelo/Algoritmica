\documentclass[12pt]{article}
\usepackage[english]{babel}
\usepackage[utf8x]{inputenc}
\usepackage{amsmath}
\usepackage{graphicx}
\usepackage[colorinlistoftodos]{todonotes}
\usepackage{listings}
\usepackage[hmargin=2cm]{geometry}
\usepackage{color} 
\definecolor{codegreen}{rgb}{0,0.6,0}
\definecolor{codegray}{rgb}{0.5,0.5,0.5}
\definecolor{codepurple}{rgb}{0.58,0,0.82}
\definecolor{backcolour}{rgb}{0.95,0.95,0.92} 
\lstdefinestyle{mystyle}{
    backgroundcolor=\color{backcolour},   
    commentstyle=\color{codegreen},
    keywordstyle=\color{magenta},
    numberstyle=\tiny\color{codegray},
    stringstyle=\color{codepurple},
    basicstyle=\footnotesize,
    breakatwhitespace=false,         
    breaklines=true,                 
    captionpos=b,                    
    keepspaces=true,                 
    numbers=left,                    
    numbersep=5pt,                  
    showspaces=false,                
    showstringspaces=false,
    showtabs=false,                  
    tabsize=2
}
\lstset{style=mystyle}
\begin{document}
\begin{titlepage}
\newcommand{\HRule}{\rule{\linewidth}{0.5mm}}
\center
\textsc{\LARGE Universidad de Granada}\\[1.5cm] % Name of your university/college
\textsc{\Large Algorítmica}\\[0.5cm] % Major heading such as course name
\textsc{\large Memoria de Prácticas}\\[0.5cm] % Minor heading such as course title
\HRule \\[0.4cm]
{ \huge \bfseries Práctica I: Eficiencia}\\[0.4cm] % Title of your document
\HRule \\[1.5cm]
\begin{minipage}{0.4\textwidth}
\begin{flushleft} \large
\emph{Autor:}\\
Antonio Gámiz Delgado \textsc{} % Your name
\end{flushleft}
\end{minipage}
~
\begin{minipage}{0.4\textwidth}
\begin{flushright} \large
\emph{} \\
\textsc{} % Supervisor's Name
\end{flushright}
\end{minipage}\\[2cm]
{\large 13 de Marzo}\\[2cm] % Date, change the \today to a set date if you want to be precise
\includegraphics{logo.png}\\[1cm]
\vfill % Fill the rest of the page with whitespace
\end{titlepage}

%\section{Análisis de la Eficiencia}


\subsection{Algoritmo de Inserción}
\begin{lstlisting}[language=C]
inline static void insercion(int T[], int num_elem)
{
  insercion_lims(T, 0, num_elem);
}

static void insercion_lims(int T[], int inicial, int final)
{
  int i, j;
  int aux;
  for (i = inicial + 1; i < final; i++) {
    j = i;
    while ((T[j] < T[j-1]) && (j > 0)) {
      aux = T[j];
      T[j] = T[j-1];
      T[j-1] = aux;
      j--;
    };
  };
}
\end{lstlisting}

Vamos a estudiar el peor caso que se le podría presentar al algoritmo de inserción, es decir, que el vector estuviera ordenado en orden inverso (de mayor a menor).

Como vemos en la línea 5, siempre se llama al método con los argumentos $0$ y $num\_elem$, por lo que a partir de ahora para nosotros, $inicial$ será $0$ y $final$ será $n$.

La mayor parte del tiempo de ejecución se emplea en el cuerpo del bucle $while$ interno. Ese trozo de código se puede acotar por una constante $a$. Por lo tanto, las líneas 15-19 se ejecutan un número de veces dependiente del bucle externo, exactamente $i$ veces (ya que estamos suponiendo que nos encontramos en el peor caso). El bucle externo se ejecuta exactamente $n$ veces, por lo que nos queda:

\[T(n)=\sum_{i=1}^{n-1}\sum_{j=1}^{i}a=a\sum_{i=0}^{n-1}\sum_{j=1}^{i}1=a\sum_{i=0}^{n-1}i=a\frac{n(n-1)}{2}\]
Por lo que vemos que $T(n)\in O(n^2)$ o cuadrático.

\subsection{Algoritmo QuickSort}
\begin{lstlisting}[language=C]

static void heapsort(int T[], int num_elem)
{
  int i;
  for (i = num_elem/2; i >= 0; i--)
    reajustar(T, num_elem, i);
  for (i = num_elem - 1; i >= 1; i--)
    {
      int aux = T[0];
      T[0] = T[i];
      T[i] = aux;
      reajustar(T, i, 0);
    }
} 

static void reajustar(int T[], int num_elem, int k)
{
  int j;
  int v;
  v = T[k];
  bool esAPO = false;
  while ((k < num_elem/2) && !esAPO)
    {
      j = k + k + 1;

      if ((j < (num_elem - 1)) && (T[j] < T[j+1])) j++;
      if (v >= T[j]) esAPO = true;
      
      T[k] = T[j];
      k = j;
    }
  T[k] = v;
}
\end{lstlisting}

Vemos que en el primer bucle de la función $heapsort$ aparece la función $reajustar$, por lo que vamos a calcular su eficiencia primero. 

Vemos que el cuerpo del bucle $while$ consume la mayor parte del tiempo de ejecución. El cuerpo de bucle se puede acotar por una constante $b$. Como vemos en la línea 22, el bucle empieza en $k$ y termina en $\frac{n}{2}$, pero $k$ no avanza de 1 en 1, sino de $2k+1$ en $2k+1$, por lo que como mucho se ejecutará $\log(n/2)$ veces. Por lo que $R(n) \in O(\log(n))$, siendo $K(n)$ la función de eficiencia de la función $reajustar$.
Una vez conocida la eficiencia de la función $reajustar$, pasamos a estudiar el primer bucle de la función $heapsort$. Va desde $n/2$ hasta $0$, así que se ejecuta $n/2$ veces, y en cada una de esas veces ejecuta la función $reajustar$, por lo que encontes su eficiencia es $\frac{nlog(n)}{2}$. El segundo bucle se ejecuta $n-1$ veces, por lo que nos queda:
\[T(n)=\frac{nlog(n)}{2}+(n-1)\]
Por lo que $T(n)\in O(n\log(n))$.

\subsection{Algoritmo de Floyd}
\begin{lstlisting}[language=C]
void Floyd(int **M, int dim)
{
	for (int k = 0; k < dim; k++)
	  for (int i = 0; i < dim;i++)
	    for (int j = 0; j < dim;j++)
	      {
				int sum = M[i][k] + M[k][j];    	
		    M[i][j] = (M[i][j] > sum) ? sum : M[i][j];
	      }
}
\end{lstlisting}

Vemos que el cuerpo del tercer bucle $while$ anidado consume la mayor parte del tiempo de ejecución, por lo que lo acotamos por una constante $a$. Fácilmente vemos que el resto de bucles va desde 0 hasta $dim$, que podemos denominar $n$ para mayor facilidad. Por lo que evidentemente tenemos que la eficiencia de este algoritmo es $an^3$, es decir, $T(n)\in O(n^3)$.

\section{Eficiencia empírica}

En esta práctica vamos a estudiar la eficiencia empírica de los siguientes algoritmos:

\begin{center}
\begin{tabular}{| c | c |}
\hline
Algoritmo & Orden de Eficiencia \\ \hline
Burbuja & O$(n^2$) \\ \hline
Inserción & O$(n^2$) \\ \hline 
Selección & O$(n^2$) \\ \hline 
Mergesort & O$(n\log(n)$) \\ \hline 
Quicksort & O$(n\log(n)$) \\ \hline 
Heapsort & O$(n\log(n)$) \\ \hline
Floyd & O$(n^3)$ \\ \hline 
Hanoi & O$(2^n)$ \\ \hline 
\hline
\end{tabular}
\end{center}

Para el estudio de la eficiencia empírica he escrito un script \footnotetext{Adjunto el código del script pero no lo explico porque creo que no es el objetivo de esta práctica.} en $Python 2.7$ para facilitarme el trabajo de la gestión de los datos. Primero, he añadido a cada programa unas líneas de código para poder medir el tiempo que ha tardado en ejecutarse. Luego he compilado cada programa con 3 opciones de optimización distintas (sin optimizar, con -O1 y con -O2) con g++. Una vez obtenidos los ejecutables, he ejecutado cada grupo de programas (me refiero al mismo algoritmo pero compilado de forma diferente) 25 veces con diferentes valores de $n$ (tamaño) para cada algoritmo y he guardado los resultados en varios archivos (los datos obtenidos se pueden ver en las tablas de abajo). Una vez hecho todo eso, he hecho un ajuste por mínimos cuadrados de los datos según su orden de eficiencia:

\begin{enumerate}
\item $O(n^2)$: ajustada por la función $f(x)=ax^2+bx+c$
\item $O(n\log(n)$: ajustada por la función $f(x)=a\log(bn)+c$
\item $O(n^3)$: ajustada por la función $f(x)=ax^3+bx^2+cx+d$
\item $O(2^n)$: ajustada por la función $f(x)=a2^{bn}+c$
\end{enumerate}

Los valores obtenidos para los coeficientes $(a,b,c,d)$ aparecen en la legenda de las gráficas según la opción de optimización usada.

Las gráficas que aparece sóla en cada página (la que hay debajo de las 3 pequeñas) muestra como varía la eficiencia del algoritmo en función de la opción de optimización usada.

A continuación muestro todos los datos obtenidos junto a las gráficas de los algoritmos expuestos arriba:

\newpage
\subsection{Algoritmo burbuja}
\begin{center}
\includegraphics[width=.4\textwidth]{../graficos/burbuja/burbuja.png}
\includegraphics[width=.4\textwidth]{../graficos/burbuja/burbuja_O1.png}
\includegraphics[width=.4\textwidth]{../graficos/burbuja/burbuja_O2.png}
\end{center}
\begin{center}
\includegraphics[width=.6\textwidth]{../graficos/burbuja/burbuja_juntas.png}
\end{center}
\begin{center}
\begin{tabular}{| c | c | c | c |}
\hline
\textbf{N} & \textbf{O-} & \textbf{O1} & \textbf{O2} \\ \hline
500 & 0.000848284 & 0.000504565 & 0.000444922 \\ \hline
1000 & 0.00155837 & 0.00170106 & 0.00172926 \\ \hline
1500 & 0.00375233 & 0.00372675 & 0.00348126 \\ \hline
2000 & 0.00625636 & 0.00654731 & 0.00675332 \\ \hline
2500 & 0.0102962 & 0.0111093 & 0.00993443 \\ \hline
3000 & 0.0158037 & 0.0156202 & 0.0159983 \\ \hline
3500 & 0.0211397 & 0.0220235 & 0.021973 \\ \hline
4000 & 0.0292756 & 0.0298859 & 0.0299241 \\ \hline
4500 & 0.0377767 & 0.0425379 & 0.0370894 \\ \hline
5000 & 0.0472146 & 0.0575588 & 0.0466534 \\ \hline
5500 & 0.0585254 & 0.0650413 & 0.0596688 \\ \hline
6000 & 0.0734717 & 0.0795291 & 0.0789712 \\ \hline
6500 & 0.0875871 & 0.0864143 & 0.0938704 \\ \hline
7000 & 0.101356 & 0.106416 & 0.10351 \\ \hline
7500 & 0.119825 & 0.117833 & 0.118222 \\ \hline
8000 & 0.13685 & 0.142887 & 0.13587 \\ \hline
8500 & 0.158619 & 0.156353 & 0.154324 \\ \hline
9000 & 0.177794 & 0.180205 & 0.179462 \\ \hline
9500 & 0.199033 & 0.208484 & 0.197654 \\ \hline
10000 & 0.225543 & 0.224484 & 0.224071 \\ \hline
10500 & 0.249815 & 0.252788 & 0.247328 \\ \hline
11000 & 0.279319 & 0.277287 & 0.284709 \\ \hline
11500 & 0.30519 & 0.305155 & 0.305884 \\ \hline
12000 & 0.335826 & 0.334685 & 0.339147 \\ \hline
12500 & 0.368191 & 0.361129 & 0.365015 \\ \hline
\hline
\end{tabular}
\end{center}
\newpage
\subsection{Algoritmo insercion}
\begin{center}
\includegraphics[width=.4\textwidth]{../graficos/insercion/insercion.png}
\includegraphics[width=.4\textwidth]{../graficos/insercion/insercion_O1.png}
\includegraphics[width=.4\textwidth]{../graficos/insercion/insercion_O2.png}
\end{center}
\begin{center}
\includegraphics[width=.6\textwidth]{../graficos/insercion/insercion_juntas.png}
\end{center}
\begin{center}
\begin{tabular}{| c | c | c | c |}
\hline
\textbf{N} & \textbf{O-} & \textbf{O1} & \textbf{O2} \\ \hline
500 & 0.000280816 & 0.000233787 & 0.000223985 \\ \hline
1000 & 0.000985635 & 0.000910265 & 0.00100876 \\ \hline
1500 & 0.00213929 & 0.00200899 & 0.00204243 \\ \hline
2000 & 0.00369846 & 0.00377572 & 0.00449 \\ \hline
2500 & 0.00592036 & 0.00560895 & 0.00585753 \\ \hline
3000 & 0.00822953 & 0.00840262 & 0.00825681 \\ \hline
3500 & 0.0120226 & 0.0115001 & 0.0129874 \\ \hline
4000 & 0.0148672 & 0.0151695 & 0.0168938 \\ \hline
4500 & 0.0193506 & 0.0185337 & 0.0193503 \\ \hline
5000 & 0.0227202 & 0.0235869 & 0.0230816 \\ \hline
5500 & 0.0272918 & 0.0275996 & 0.0280322 \\ \hline
6000 & 0.0333669 & 0.0314578 & 0.0388916 \\ \hline
6500 & 0.0395249 & 0.0369619 & 0.0393369 \\ \hline
7000 & 0.0438121 & 0.0443715 & 0.0442169 \\ \hline
7500 & 0.0510923 & 0.0534808 & 0.0519345 \\ \hline
8000 & 0.0577335 & 0.0668524 & 0.0582985 \\ \hline
8500 & 0.0664888 & 0.0740554 & 0.0659169 \\ \hline
9000 & 0.0721457 & 0.0740537 & 0.0733521 \\ \hline
9500 & 0.0804506 & 0.0808475 & 0.081258 \\ \hline
10000 & 0.0889614 & 0.089042 & 0.0897421 \\ \hline
10500 & 0.098722 & 0.0999622 & 0.0988874 \\ \hline
11000 & 0.10898 & 0.111618 & 0.108517 \\ \hline
11500 & 0.120852 & 0.118909 & 0.125629 \\ \hline
12000 & 0.129697 & 0.128718 & 0.137064 \\ \hline
12500 & 0.141071 & 0.140754 & 0.141494 \\ \hline
\hline
\end{tabular}
\end{center}

\newpage
\subsection{Algoritmo seleccion}
\begin{center}
\includegraphics[width=.4\textwidth]{../graficos/seleccion/seleccion.png}
\includegraphics[width=.4\textwidth]{../graficos/seleccion/seleccion_O1.png}
\includegraphics[width=.4\textwidth]{../graficos/seleccion/seleccion_O2.png}
\end{center}
\begin{center}
\includegraphics[width=.6\textwidth]{../graficos/seleccion/seleccion_juntas.png}
\end{center}
\begin{center}
\begin{tabular}{| c | c | c | c |}
\hline
\textbf{N} & \textbf{O-} & \textbf{O1} & \textbf{O2} \\ \hline
500 & 0.0314449 & 0.0314842 & 0.0317225 \\ \hline
1000 & 0.119021 & 0.126228 & 0.11633 \\ \hline
1500 & 0.260984 & 0.267642 & 0.258436 \\ \hline
2000 & 0.457069 & 0.461125 & 0.454904 \\ \hline
2500 & 0.00744243 & 0.00693814 & 0.0068266 \\ \hline
3000 & 0.00998253 & 0.0103105 & 0.0104942 \\ \hline
3500 & 0.0138713 & 0.0136245 & 0.0144952 \\ \hline
4000 & 0.0186035 & 0.0182824 & 0.0187636 \\ \hline
4500 & 0.0234744 & 0.0224572 & 0.0237714 \\ \hline
5000 & 0.0277004 & 0.0292558 & 0.0300571 \\ \hline
5500 & 0.0345149 & 0.0354603 & 0.035354 \\ \hline
6000 & 0.0478933 & 0.0406127 & 0.0407739 \\ \hline
6500 & 0.0520387 & 0.0483603 & 0.0474503 \\ \hline
7000 & 0.0560279 & 0.0554717 & 0.0550182 \\ \hline
7500 & 0.0618687 & 0.067223 & 0.0635355 \\ \hline
8000 & 0.0725331 & 0.0711132 & 0.0718062 \\ \hline
8500 & 0.0815797 & 0.0811896 & 0.0813514 \\ \hline
9000 & 0.0909342 & 0.0917913 & 0.0928178 \\ \hline
9500 & 0.103387 & 0.100633 & 0.100902 \\ \hline
10000 & 0.110117 & 0.11207 & 0.127083 \\ \hline
10500 & 0.12314 & 0.124775 & 0.125395 \\ \hline
11000 & 0.134518 & 0.135545 & 0.135101 \\ \hline
11500 & 0.151427 & 0.146785 & 0.146842 \\ \hline
12000 & 0.158935 & 0.163465 & 0.159707 \\ \hline
12500 & 0.174672 & 0.174286 & 0.174868 \\ \hline
\hline
\end{tabular}
\end{center}

\newpage
\subsection{Algoritmo mergesort}
\begin{center}
\includegraphics[width=.4\textwidth]{../graficos/mergesort/mergesort.png}
\includegraphics[width=.4\textwidth]{../graficos/mergesort/mergesort_O1.png}
\includegraphics[width=.4\textwidth]{../graficos/mergesort/mergesort_O2.png}
\end{center}
\begin{center}
\includegraphics[width=.6\textwidth]{../graficos/mergesort/mergesort_juntas.png}
\end{center}
\begin{center}
\begin{tabular}{| c | c | c | c |}
\hline
\textbf{N} & \textbf{O-} & \textbf{O1} & \textbf{O2} \\ \hline
500 & 4.6913e-05 & 4.2538e-05 & 4.3779e-05 \\ \hline
1000 & 0.000104759 & 0.000101012 & 0.000102685 \\ \hline
1500 & 0.000197928 & 0.000202091 & 0.000197977 \\ \hline
2000 & 0.000230556 & 0.000230267 & 0.000229837 \\ \hline
2500 & 0.000320995 & 0.000321507 & 0.000324846 \\ \hline
3000 & 0.000441303 & 0.000454994 & 0.000459037 \\ \hline
3500 & 0.000411042 & 0.000422831 & 0.000415593 \\ \hline
4000 & 0.000511027 & 0.000494096 & 0.000499695 \\ \hline
4500 & 0.000612995 & 0.000592159 & 0.000609127 \\ \hline
5000 & 0.00069516 & 0.000701389 & 0.000693268 \\ \hline
5500 & 0.000804711 & 0.000800892 & 0.000801765 \\ \hline
6000 & 0.000939736 & 0.0009096 & 0.000922795 \\ \hline
6500 & 0.000810641 & 0.000809706 & 0.000812514 \\ \hline
7000 & 0.000890017 & 0.000891093 & 0.000896917 \\ \hline
7500 & 0.000985485 & 0.000987166 & 0.000983415 \\ \hline
8000 & 0.00110324 & 0.00107382 & 0.00107462 \\ \hline
8500 & 0.00116999 & 0.00116523 & 0.00117428 \\ \hline
9000 & 0.00127648 & 0.00127043 & 0.00127371 \\ \hline
9500 & 0.00138226 & 0.00137334 & 0.00139195 \\ \hline
10000 & 0.00150468 & 0.00148788 & 0.00148267 \\ \hline
10500 & 0.001526 & 0.001525 & 0.001523 \\ \hline
11000 & 0.001634 & 0.001625 & 0.002057 \\ \hline
11500 & 0.001889 & 0.001821 & 0.00186 \\ \hline
12000 & 0.002311 & 0.001866 & 0.002124 \\ \hline
12500 & 0.002148 & 0.002275 & 0.002152 \\ \hline
\hline
\end{tabular}
\end{center}

\newpage
\subsection{Algoritmo quicksort}
\begin{center}
\includegraphics[width=.4\textwidth]{../graficos/quicksort/quicksort.png}
\includegraphics[width=.4\textwidth]{../graficos/quicksort/quicksort_O1.png}
\includegraphics[width=.4\textwidth]{../graficos/quicksort/quicksort_O2.png}
\end{center}
\begin{center}
\includegraphics[width=.6\textwidth]{../graficos/quicksort/quicksort_juntas.png}
\end{center}
\begin{center}
\begin{tabular}{| c | c | c | c |}
\hline
\textbf{N} & \textbf{O-} & \textbf{O1} & \textbf{O2} \\ \hline
500 & 3.2332e-05 & 3.7642e-05 & 0.000103603 \\ \hline
1000 & 7.036e-05 & 8.0497e-05 & 0.000222952 \\ \hline
1500 & 0.000113397 & 0.000131637 & 0.000361161 \\ \hline
2000 & 0.000154328 & 0.000176116 & 0.000484841 \\ \hline
2500 & 0.000200578 & 0.00023071 & 0.000631374 \\ \hline
3000 & 0.000240971 & 0.000276572 & 0.000763986 \\ \hline
3500 & 0.000284638 & 0.000360803 & 0.000897255 \\ \hline
4000 & 0.000333301 & 0.000420274 & 0.00107918 \\ \hline
4500 & 0.000373817 & 0.000525212 & 0.001179 \\ \hline
5000 & 0.000419683 & 0.000665588 & 0.00133037 \\ \hline
5500 & 0.000463525 & 0.000837868 & 0.00146456 \\ \hline
6000 & 0.000521563 & 0.000963575 & 0.00166409 \\ \hline
6500 & 0.000557223 & 0.00101092 & 0.0017721 \\ \hline
7000 & 0.000617823 & 0.00132214 & 0.00197305 \\ \hline
7500 & 0.000920243 & 0.00140165 & 0.00211571 \\ \hline
8000 & 0.000714884 & 0.00180813 & 0.00226463 \\ \hline
8500 & 0.000766829 & 0.00193638 & 0.00243405 \\ \hline
9000 & 0.000817047 & 0.00206904 & 0.00260327 \\ \hline
9500 & 0.000909397 & 0.00218516 & 0.00275356 \\ \hline
10000 & 0.000957013 & 0.00232035 & 0.00288264 \\ \hline
10500 & 0.00108144 & 0.00242214 & 0.00302737 \\ \hline
11000 & 0.00115832 & 0.00257706 & 0.00429314 \\ \hline
11500 & 0.00122584 & 0.00265797 & 0.00448855 \\ \hline
12000 & 0.00127122 & 0.00277936 & 0.0046674 \\ \hline
12500 & 0.00135047 & 0.00373325 & 0.00497125 \\ \hline
\hline
\end{tabular}
\end{center}

\newpage
\subsection{Algoritmo heapsort}
\begin{center}
\includegraphics[width=.4\textwidth]{../graficos/heapsort/heapsort.png}
\includegraphics[width=.4\textwidth]{../graficos/heapsort/heapsort_O1.png}
\includegraphics[width=.4\textwidth]{../graficos/heapsort/heapsort_O2.png}
\end{center}
\begin{center}
\includegraphics[width=.6\textwidth]{../graficos/heapsort/heapsort_juntas.png}
\end{center}
\begin{center}
\begin{tabular}{| c | c | c | c |}
\hline
\textbf{N} & \textbf{O-} & \textbf{O1} & \textbf{O2} \\ \hline
500 & 4.5886e-05 & 4.7776e-05 & 5.2733e-05 \\ \hline
1000 & 9.3484e-05 & 0.000105841 & 0.000115869 \\ \hline
1500 & 0.000155393 & 0.000165102 & 0.000180243 \\ \hline
2000 & 0.0002156 & 0.000226943 & 0.000250445 \\ \hline
2500 & 0.00027561 & 0.000293035 & 0.000318913 \\ \hline
3000 & 0.000335879 & 0.000370002 & 0.00038954 \\ \hline
3500 & 0.000400699 & 0.000423861 & 0.000472134 \\ \hline
4000 & 0.000462112 & 0.000537971 & 0.00053706 \\ \hline
4500 & 0.000574792 & 0.000613159 & 0.000681049 \\ \hline
5000 & 0.00062758 & 0.000688875 & 0.000936752 \\ \hline
5500 & 0.000696955 & 0.000764817 & 0.00107965 \\ \hline
6000 & 0.000771178 & 0.000843093 & 0.00137774 \\ \hline
6500 & 0.0008542 & 0.000930073 & 0.00153718 \\ \hline
7000 & 0.000948121 & 0.00100329 & 0.0016506 \\ \hline
7500 & 0.000989718 & 0.00108493 & 0.00176835 \\ \hline
8000 & 0.00106711 & 0.00118179 & 0.00193712 \\ \hline
8500 & 0.00113612 & 0.00124409 & 0.00206274 \\ \hline
9000 & 0.00121429 & 0.00132679 & 0.00220556 \\ \hline
9500 & 0.00128797 & 0.00140478 & 0.00231064 \\ \hline
10000 & 0.00136625 & 0.00149198 & 0.00250894 \\ \hline
10500 & 0.00144845 & 0.00157132 & 0.00262056 \\ \hline
11000 & 0.00151039 & 0.00166773 & 0.00271144 \\ \hline
11500 & 0.00158849 & 0.00174876 & 0.00295809 \\ \hline
12000 & 0.00166249 & 0.00188693 & 0.00302842 \\ \hline
12500 & 0.00174171 & 0.00191495 & 0.00311223 \\ \hline
\hline
\end{tabular}
\end{center}

\newpage
\subsection{Algoritmo floyd}
\begin{center}
\includegraphics[width=.4\textwidth]{../graficos/floyd/floyd.png}
\includegraphics[width=.4\textwidth]{../graficos/floyd/floyd_O1.png}
\includegraphics[width=.4\textwidth]{../graficos/floyd/floyd_O2.png}
\end{center}
\begin{center}
\includegraphics[width=.6\textwidth]{../graficos/floyd/floyd_juntas.png}
\end{center}
\begin{center}
\begin{tabular}{| c | c | c | c |}
\hline
\textbf{N} & \textbf{O-} & \textbf{O1} & \textbf{O2} \\ \hline
5 & 2e-06 & 1e-06 & 2e-06 \\ \hline
10 & 7e-06 & 7e-06 & 7e-06 \\ \hline
15 & 2.1e-05 & 2.1e-05 & 2.1e-05 \\ \hline
20 & 4.6e-05 & 5e-05 & 4.9e-05 \\ \hline
25 & 8.3e-05 & 8.9e-05 & 8.7e-05 \\ \hline
30 & 0.000145 & 0.000149 & 0.000146 \\ \hline
35 & 0.000227 & 0.000244 & 0.00023 \\ \hline
40 & 0.000335 & 0.000377 & 0.000336 \\ \hline
45 & 0.000455 & 0.000512 & 0.000455 \\ \hline
50 & 0.000636 & 0.000731 & 0.000642 \\ \hline
55 & 0.000834 & 0.000952 & 0.000841 \\ \hline
60 & 0.001134 & 0.001233 & 0.001076 \\ \hline
65 & 0.001432 & 0.001545 & 0.001376 \\ \hline
70 & 0.001715 & 0.001924 & 0.001768 \\ \hline
75 & 0.002059 & 0.00235 & 0.002205 \\ \hline
80 & 0.002596 & 0.002852 & 0.002645 \\ \hline
85 & 0.003052 & 0.003377 & 0.003118 \\ \hline
90 & 0.003596 & 0.003975 & 0.004017 \\ \hline
95 & 0.004309 & 0.00473 & 0.004775 \\ \hline
100 & 0.004978 & 0.005552 & 0.005583 \\ \hline
105 & 0.005801 & 0.005681 & 0.006403 \\ \hline
110 & 0.006597 & 0.007624 & 0.009087 \\ \hline
115 & 0.007513 & 0.007536 & 0.008881 \\ \hline
120 & 0.008491 & 0.008468 & 0.009428 \\ \hline
125 & 0.009594 & 0.009587 & 0.010736 \\ \hline
\hline
\end{tabular}
\end{center}

\newpage
\subsection{Algoritmo hanoi}
\begin{center}
\includegraphics[width=.4\textwidth]{../graficos/hanoi/hanoi.png}
\includegraphics[width=.4\textwidth]{../graficos/hanoi/hanoi_O1.png}
\includegraphics[width=.4\textwidth]{../graficos/hanoi/hanoi_O2.png}
\end{center}
\begin{center}
\includegraphics[width=.6\textwidth]{../graficos/hanoi/hanoi_juntas.png}
\end{center}
\begin{center}
\begin{tabular}{| c | c | c | c |}
\hline
\textbf{N} & \textbf{O-} & \textbf{O1} & \textbf{O2} \\ \hline
0.5 & 9.3e-08 & 1.43e-07 & 1.28e-07 \\ \hline
1.0 & 1.71e-07 & 1.6e-07 & 2.97e-07 \\ \hline
1.5 & 1.66e-07 & 1.99e-07 & 1.53e-07 \\ \hline
2.0 & 2.04e-07 & 2.7e-07 & 2.29e-07 \\ \hline
2.5 & 1.41e-07 & 1.45e-07 & 2.08e-07 \\ \hline
3.0 & 2.51e-07 & 2.92e-07 & 3.47e-07 \\ \hline
3.5 & 2.54e-07 & 2.86e-07 & 3.04e-07 \\ \hline
4.0 & 3.91e-07 & 4.69e-07 & 4.54e-07 \\ \hline
4.5 & 3.62e-07 & 4.08e-07 & 4.67e-07 \\ \hline
5.0 & 5.49e-07 & 6.58e-07 & 7.21e-07 \\ \hline
5.5 & 5.26e-07 & 6.2e-07 & 6.98e-07 \\ \hline
6.0 & 8.55e-07 & 9.13e-07 & 1.057e-06 \\ \hline
6.5 & 7.52e-07 & 8.07e-07 & 1.037e-06 \\ \hline
7.0 & 1.16e-06 & 1.374e-06 & 1.563e-06 \\ \hline
7.5 & 1.199e-06 & 1.497e-06 & 1.638e-06 \\ \hline
8.0 & 2.104e-06 & 2.197e-06 & 2.487e-06 \\ \hline
8.5 & 2.187e-06 & 2.384e-06 & 2.744e-06 \\ \hline
9.0 & 3.666e-06 & 3.97e-06 & 4.553e-06 \\ \hline
9.5 & 3.732e-06 & 3.788e-06 & 4.352e-06 \\ \hline
10.0 & 6.565e-06 & 7.148e-06 & 8.174e-06 \\ \hline
10.5 & 6.523e-06 & 7.279e-06 & 8.151e-06 \\ \hline
11.0 & 1.242e-05 & 1.3911e-05 & 1.5101e-05 \\ \hline
11.5 & 1.2321e-05 & 1.365e-05 & 1.5014e-05 \\ \hline
12.0 & 2.4334e-05 & 2.6554e-05 & 2.9694e-05 \\ \hline
12.5 & 2.4104e-05 & 2.9556e-05 & 2.9656e-05 \\ \hline
\hline
\end{tabular}
\end{center}



\section{Algoritmos de búsqueda}
\begin{center}
\includegraphics[scale=0.6]{../graficos/ordenacion/ordenacion.png}
\end{center}
En la gráfica de arriba he representado la eficiencia empírica de todos los algoritmos de ordenación estudiados más arriba. Observando el gráfico, vemos la clara superioridad de los algoritmos quicksort, heapsort y mergesort. 

Vemos, también, que para tamaños entre 5000 y 6000, el algoritmo de selección es más lento que el de la burbuja, pero para tamaños superiores, el algoritmo de selección es notablemente más rápido (esto se debe a que el valor del UMBRAL se encuentra entre esos valores). Además observamos que el algoritmo de inserción siempre es más rápido que el de selección y el de la burbuja.

\begin{center}
\includegraphics[scale=0.5]{../graficos/ordenacion/ordenacion_sort_only.png}
\end{center}

Debido a que en la anterior imagen no podemos comparar adecuadamente los algoritmos quicksort, mergesort y heapsort, he generado otra gráfica con ellos solos para una mejor visualización. 

Vemos que para tamaños menores a 2000 el algoritmo mergesort es más lento que los demás, pero a partir de ese tamaño se hace evidente que es el mejor algoritmo de ordenación de todos los vistos aquí.

\section{Variación del ajuste según la función}

\includegraphics[scale=0.9]{../graficos/ordenacion/ajuste.png}

En la gráfica de arriba he realizado el ajuste correspondiente al algoritmo de la burbuja optimizado con la opción $O2$, pero esta vez ajustándolo a otras funciones. Como vemos, la calidad del ajuste es pésima, la función $f(x)=x^3$ sale tan alejadda de los datos que ni si quiere se muestra en el gráfico, y la función logarítima no se acerca mucho tampoco. Sin embargo, como vemos, la función $x^2$, se ciñe perfectamente a los datos tomados. Aquí se ve la importancia de realizar un buen estudio teórico de la eficiencia antes de realizar el estudio empírico.

\end{document}
