\section{Análisis de la Eficiencia}


\subsection{Algoritmo de Inserción}
\begin{lstlisting}[language=C]
inline static void insercion(int T[], int num_elem)
{
  insercion_lims(T, 0, num_elem);
}

static void insercion_lims(int T[], int inicial, int final)
{
  int i, j;
  int aux;
  for (i = inicial + 1; i < final; i++) {
    j = i;
    while ((T[j] < T[j-1]) && (j > 0)) {
      aux = T[j];
      T[j] = T[j-1];
      T[j-1] = aux;
      j--;
    };
  };
}
\end{lstlisting}

Estudiemos el peor caso que se le podría presentar al algoritmo de inserción: que el vector estuviera ordenado en orden inverso (de mayor a menor).

Como vemos en la línea 5, siempre se llama al método con los argumentos $0$ y $num\_elem$, por lo que a partir de ahora para nosotros, $inicial$ será $0$ y $final$ será $n$.

La mayor parte del tiempo de ejecución se emplea en el cuerpo del bucle $while$ interno. Ese trozo de código se puede acotar por una constante $a$. Por lo tanto, las líneas 15-19 se ejecutan un número de veces dependiente del bucle externo, exactamente $i$ veces (ya que estamos suponiendo que nos encontramos en el peor caso). El bucle externo se ejecuta exactamente $n$ veces, por lo que nos queda:

\[T(n)=\sum_{i=1}^{n-1}\sum_{j=1}^{i}a=a\sum_{i=0}^{n-1}\sum_{j=1}^{i}1=a\sum_{i=0}^{n-1}i=a\frac{n(n-1)}{2}\]
Por lo que vemos que $T(n)\in O(n^2)$, es cuadrático.

\newpage

\subsection{Algoritmo Selección}

\begin{lstlisting}[language=C]
void seleccion(int T[], int num_elem)
{
  seleccion_lims(T, 0, num_elem);
}

static void seleccion_lims(int T[], int inicial, int final)
{
  int i, j, indice_menor;
  int menor, aux;
  for (i = inicial; i < final - 1; i++) {
    indice_menor = i;
    menor = T[i];
    for (j = i; j < final; j++)
      if (T[j] < menor) {
			indice_menor = j;
			menor = T[j];
      }
    aux = T[i];
    T[i] = T[indice_menor];
    T[indice_menor] = aux;
  };
}
\end{lstlisting}

Vamos a estudiar el peor caso que se le podría presentar al algoritmo de selección, es decir, que el vector estuviera ordenado a la inversa (de mayor a menor).

Como vemos en la línea 3, siempre se llama al método con los argumentos $0$ y $num\_elem$, por lo que a partir de ahora para nosotros, $inicial$ será $0$ y $final$ será $n$.

Las líneas 8 y 9 son de orden $O(1)$ por lo que las descartamos. Vemos que el prácticamente todo el tiempo de ejecución es consumido por los dos bucles $for$ de las líneas 10 y 13. El primer bucle se ejecuta $n$ veces, mientras que el bucle interior se ejecutará $n-i$ veces por cada iteración del bucle anterior. Por lo que tenemos:

\[
\sum_{i=0}^{n-1}(n-i-1)=\sum_{i=0}^{n-1}(n-1)-\sum_{i=0}^{n-1}i=(n-1)n-\frac{n(n+1)}{2}=\frac{n^2}{2}-\frac{n}{2}
\]

Por lo que tenemos que, $T(n)\in O(n^2)$, donde $T(n)$ es la función de eficiencia del algoritmo de selección.

\newpage 

\subsection{Algoritmo Heapsort}

\begin{lstlisting}[language=C]

static void heapsort(int T[], int num_elem)
{
  int i;
  for (i = num_elem/2; i >= 0; i--)
    reajustar(T, num_elem, i);
  for (i = num_elem - 1; i >= 1; i--)
    {
      int aux = T[0];
      T[0] = T[i];
      T[i] = aux;
      reajustar(T, i, 0);
    }
} 

static void reajustar(int T[], int num_elem, int k)
{
  int j;
  int v;
  v = T[k];
  bool esAPO = false;
  while ((k < num_elem/2) && !esAPO)
    {
      j = k + k + 1;

      if ((j < (num_elem - 1)) && (T[j] < T[j+1])) j++;
      if (v >= T[j]) esAPO = true;
      
      T[k] = T[j];
      k = j;
    }
  T[k] = v;
}
\end{lstlisting}

Vemos que en el primer bucle de la función $heapsort$ aparece la función $reajustar$, por lo que vamos a calcular su eficiencia primero. 

El cuerpo del bucle $while$ consume la mayor parte del tiempo de ejecución, y se puede acotar por una constante $b$. Como se puede observar en la línea 22, el bucle empieza en $k$ y termina en $\frac{n}{2}$, pero $k$ no avanza de 1 en 1, sino de $2k+1$ en $2k+1$, por lo que como mucho se ejecutará $\log(n/2)$ veces. Así, $R(n) \in O(\log(n))$, siendo $K(n)$ la función de eficiencia de la función $reajustar$.
Una vez conocida la eficiencia de la función $reajustar$, pasamos a estudiar el primer bucle de la función $heapsort$. Va desde $n/2$ hasta $0$, luego se ejecuta $n/2$ veces, y cada una de esas veces ejecuta la función $reajustar$, por lo que entonces su eficiencia es $\frac{nlog(n)}{2}$. El segundo bucle se ejecuta $n-1$ veces, quedando:
\[T(n)=\frac{nlog(n)}{2}+(n-1)\]
Por lo que $T(n)\in O(n\log(n))$.

\subsection{Algoritmo de Floyd}

\begin{lstlisting}[language=C]
void Floyd(int **M, int dim)
{
	for (int k = 0; k < dim; k++)
	  for (int i = 0; i < dim;i++)
	    for (int j = 0; j < dim;j++)
	      {
				int sum = M[i][k] + M[k][j];    	
		    M[i][j] = (M[i][j] > sum) ? sum : M[i][j];
	      }
}
\end{lstlisting}

Como se puede observar, el cuerpo del tercer bucle $while$ anidado consume la mayor parte del tiempo de ejecución, por lo que lo acotamos por una constante $a$. El resto de bucles va desde 0 hasta $dim$, que podemos denominar $n$ para mayor facilidad. Por lo que evidentemente tenemos que la eficiencia de este algoritmo es $an^3$, es decir, $T(n)\in O(n^3)$.


\subsection{Algoritmo de Hanoi}
\begin{lstlisting}[language=C]
void hanoi (int M, int i, int j)
{
  if (M > 0)
    {
      hanoi(M-1, i, 6-i-j);
      hanoi (M-1, 6-i-j, j);
  }
}

\end{lstlisting}

De la clase de teoría sabemos que la función de Hanoi tiene como función de eficienciaa:
\[
T(n)=\left\lbrace
\begin{array}{cc}
2T(\frac{n}{2})+1 & n \geq 1 \\
1 & n=1
\end{array}
\right.
\]

\[
\begin{array}{cc}
T(n)=2\overbrace{T(n-1)}^{2T(n-2)+1}+1 & n > 1 \\
T(n)=2^2T(n-2)+2+1 & n > 2 \\
T(n)=2^iT(n-i)+(2^{i-1}+\dots+2^2+2+1) & n > i
\end{array}
\]

Por lo que si tomamos $i=n-1$:

\[T(n)=2^{n-1}+2^{n-2}+\dots+2^2+2+1=\frac{2^{n-1}2-1}{2-1}=2^n-1\]

Por lo que tenemos que $T(n)\in O(2^n)$.