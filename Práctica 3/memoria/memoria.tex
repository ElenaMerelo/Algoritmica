\documentclass[11pt]{article}
\usepackage[english]{babel}
\usepackage[utf8x]{inputenc}
\usepackage{amsmath}
\usepackage{graphicx}
\usepackage[colorinlistoftodos]{todonotes}
\usepackage{listings}
\usepackage[hmargin=2cm]{geometry}
\usepackage{color} 
\definecolor{codegreen}{rgb}{0,0.6,0}
\definecolor{codegray}{rgb}{0.5,0.5,0.5}
\definecolor{codepurple}{rgb}{0.58,0,0.82}
\definecolor{backcolour}{rgb}{0.95,0.95,0.92} 
\lstdefinestyle{mystyle}{
    backgroundcolor=\color{backcolour},   
    commentstyle=\color{codegreen},
    keywordstyle=\color{magenta},
    numberstyle=\tiny\color{blue},
    stringstyle=\color{codepurple},
    basicstyle=\footnotesize,
    breakatwhitespace=false,         
    breaklines=true,                 
    captionpos=b,                    
    keepspaces=true,                 
    numbers=left,                    
    numbersep=5pt,                  
    showspaces=false,                
    showstringspaces=false,
    showtabs=false,                  
    tabsize=2
}
\lstset{style=mystyle}
\begin{document}
\begin{titlepage}
\newcommand{\HRule}{\rule{\linewidth}{0.5mm}}
\center
\textsc{\LARGE Universidad de Granada}\\[1.5cm] % Name of your university/college
\textsc{\Large Algorítmica}\\[0.5cm] % Major heading such as course name
\textsc{\large Memoria de Prácticas}\\[0.5cm] % Minor heading such as course title
\HRule \\[0.4cm]
{ \huge \bfseries Práctica III: Greedy}\\[0.4cm] % Title of your document
\HRule \\[1.5cm]
\begin{minipage}{0.4\textwidth}
\begin{flushleft} \large
\emph{Autores:}\\
Antonio Gámiz Delgado\textsc{} % Your name
\end{flushleft}
\end{minipage}
~
\begin{minipage}{0.4\textwidth}
\begin{flushright} \large
\emph{} \\
\textsc{} % Supervisor's Name
\end{flushright}
\end{minipage}\\[2cm]
{\large \today}\\[2cm] % Date, change the \today to a set date if you want to be precise
\includegraphics{logo.png}\\[1cm]
\vfill % Fill the rest of the page with whitespace
\end{titlepage}

\section{Problema}

Un electricista necesita hacer $n$ reparaciones urgentes, y sabe de antemano el tiempo que le va a llevar cada una de ellas: en la tarea $i$-ésima tardará $t_i$ minutos. Como en su empresa le pagan dependiendo de la satisfacción del cliente, necesita decidir el orden en el que atenderá los avisos para minimizar el tiempo medio de atención de los clientes (desde el inicio hasta que su reparación sea efectuada).

\section{Diseño del algoritmo Greedy}

A continuación voy a explicar el algoritmo que he diseñado para resolver este problema, primero voy a exponer de las partes (más bien funciones) de las que se compone y luego explicaré como trabaja.

\begin{lstlisting}[language=C++, caption=Función obejtivo]
vector<node> greedy(graph &g, int begin)
{
	vector<node> nodes=g.get_nodes();
	vector<node> solution;
	
	int index_nextNode;
	struct node current_node=nodes[begin];
	bool first_time=true;
	solution.push_back(current_node);
	
	struct node nextNode;
	first_time=true;
	
		while(solution.size() < nodes.size())
		{
		do
		{
			index_nextNode=nodeHeuristic(g, current_node, (first_time)? 0:g.get_weight(current_node.label, nextNode.label), solution);
			if( index_nextNode==-1 ) return vector<node>();
			
			nextNode=nodes[index_nextNode];
			
			first_time=false;
		} while( cycle( solution, nextNode ) );
	
		solution.push_back(nextNode);
		current_node=nextNode;
	}

	return solution;
}
\end{lstlisting}

Esta es la función que nos interesa optimizar. Recibe por referencia un grafo $g$ (clase implementada en $graph.c$, representación interna: usando la matriz de adyacencia) y un índice natural $begin$ que indica desde que nodo queremos empezar el algoritmo (ya que la solución de este depende del nodo inicial elegido). Lo que hace el algoritmo está más o menos claro, empieza desde el nodo $nodes[begin]$ ($nodes$ contiene los nodos del grafo), y lo añadimos a la solución. Luego vamos seleccionando el nodo tal que el camino que forme con el anterior tenga peso mínimo, haciendo uso de las funciones que describo a continuación.

\begin{lstlisting}[language=C++, caption=Función para validar la solución]
bool cycle(vector<node>& v, struct node new_node){ return is_in(new_node.label, v); }
\end{lstlisting}

Como la nota indica, esta función se encarga de validar la solución, es decir, de comprobar que el nodo que queremos añadir a la solución no forme ciclo. Esto se consigue comprando que el vector a añadir \textcolor{red}{no esté} ya en la solución, ya que si estuviera, formaría ciclo. Para ello, uso una función auxiliar $is\_in$ que comprueba si el nodo $new_node$ está en el vector $v$.

\begin{lstlisting}[language=C++, caption=Función de selección]
int nodeHeuristic(graph &g, struct node n, float min_bound, vector<node> taken){
	float min=LONG_MAX;
	int index_min=-1;
	float current_weight;
	
	for(int i=0; i<g.size(); i++)
	{
		current_weight=g.get_weight(i, n.label);
		if( current_weight > 0 && current_weight < min && current_weight > min_bound && !is_in(i, taken) )
			{
				min=current_weight;
				index_min=i;
			}        
		}
	
		return index_min;
}
\end{lstlisting}

Esta función se encarga de, dado un grafo $g$, y un nodo $n$, buscar el nodo con índice $index\_min$ (el índice con el que es representado en la matriz de adyacencia) tal que el camino de $n$ a ese nodo sea el \textcolor{red}{mínimo}.

Además, si el nodo con índice $index\_min$ que encuentra está ya en la solución ($vector<node> taken$) lo \textcolor{red}{desecha} y busca otro.

El parámetro $min\_bound$ es muy importante, se encarga de fijar la cota inferior del mínimo, es decir, la función $nodeHeuristic$ busca el arco en la matriz de adyacencia justamente superior al valor de $min\_bound$. Esto lo hago debido a que si la función ha sido llamada alguna vez (cosa que compruebo con la variable $bool$ $first\_time$ en $greedy$), y la función $cycle$ ha detectado que ese nodo \textcolor{red}{forma un ciclo}, entonces hay que desecharlo, por lo que para que no encuentre ese nodo otra vez, busco el valor superior.
Como vemos, si esta función no encuentra un nodo que no forme ciclo con distancia mínima, entonces devuelve -1, indicando a la función $greedy$ que el algoritmo no puede encontrar una solución.



\end{document}