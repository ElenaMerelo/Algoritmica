%%%%%%%%%%%%%%%%%%%%%%%%%%%%%%%%%%%%%%%%%
% University Assignment Title Page 
% LaTeX Template
% Version 1.0 (27/12/12)
%
% This template has been downloaded from:
% http://www.LaTeXTemplates.com
%
% Original author:
% WikiBooks (http://en.wikibooks.org/wiki/LaTeX/Title_Creation)
%
% License:
% CC BY-NC-SA 3.0 (http://creativecommons.org/licenses/by-nc-sa/3.0/)
% 
% Instructions for using this template:
% This title page is capable of being compiled as is. This is not useful for 
% including it in another document. To do this, you have two options: 
%
% 1) Copy/paste everything between \begin{document} and \end{document} 
% starting at \begin{titlepage} and paste this into another LaTeX file where you 
% want your title page.
% OR
% 2) Remove everything outside the \begin{titlepage} and \end{titlepage} and 
% move this file to the same directory as the LaTeX file you wish to add it to. 
% Then add \input{./title_page_1.tex} to your LaTeX file where you want your
% title page.
%
%%%%%%%%%%%%%%%%%%%%%%%%%%%%%%%%%%%%%%%%%
%\title{Title page with logo}
%----------------------------------------------------------------------------------------
%	PACKAGES AND OTHER DOCUMENT CONFIGURATIONS
%----------------------------------------------------------------------------------------

\documentclass[12pt]{article}
\usepackage[english]{babel}
\usepackage[utf8x]{inputenc}
\usepackage{amsmath}
\usepackage{graphicx}
\usepackage[colorinlistoftodos]{todonotes}
\usepackage{listings}


\usepackage{color}
 
\definecolor{codegreen}{rgb}{0,0.6,0}
\definecolor{codegray}{rgb}{0.5,0.5,0.5}
\definecolor{codepurple}{rgb}{0.58,0,0.82}
\definecolor{backcolour}{rgb}{0.95,0.95,0.92}
 
\lstdefinestyle{mystyle}{
    backgroundcolor=\color{backcolour},   
    commentstyle=\color{codegreen},
    keywordstyle=\color{magenta},
    numberstyle=\tiny\color{codegray},
    stringstyle=\color{codepurple},
    basicstyle=\footnotesize,
    breakatwhitespace=false,         
    breaklines=true,                 
    captionpos=b,                    
    keepspaces=true,                 
    numbers=left,                    
    numbersep=5pt,                  
    showspaces=false,                
    showstringspaces=false,
    showtabs=false,                  
    tabsize=2
}
 
\lstset{style=mystyle}


\begin{document}

\begin{titlepage}

\newcommand{\HRule}{\rule{\linewidth}{0.5mm}} % Defines a new command for the horizontal lines, change thickness here

\center % Center everything on the page
 
%----------------------------------------------------------------------------------------
%	HEADING SECTIONS
%----------------------------------------------------------------------------------------

\textsc{\LARGE Universidad de Granada}\\[1.5cm] % Name of your university/college
\textsc{\Large Algorítmica}\\[0.5cm] % Major heading such as course name
\textsc{\large Memoria de Prácticas}\\[0.5cm] % Minor heading such as course title

%----------------------------------------------------------------------------------------
%	TITLE SECTION
%----------------------------------------------------------------------------------------

\HRule \\[0.4cm]
{ \huge \bfseries Práctica I: Eficiencia}\\[0.4cm] % Title of your document
\HRule \\[1.5cm]
 
%----------------------------------------------------------------------------------------
%	AUTHOR SECTION
%----------------------------------------------------------------------------------------

\begin{minipage}{0.4\textwidth}
\begin{flushleft} \large
\emph{Autor:}\\
Antonio Gámiz Delgado \textsc{} % Your name
\end{flushleft}
\end{minipage}
~
\begin{minipage}{0.4\textwidth}
\begin{flushright} \large
\emph{} \\
\textsc{} % Supervisor's Name
\end{flushright}
\end{minipage}\\[2cm]

% If you don't want a supervisor, uncomment the two lines below and remove the section above
%\Large \emph{Author:}\\
%John \textsc{Smith}\\[3cm] % Your name

%----------------------------------------------------------------------------------------
%	DATE SECTION
%----------------------------------------------------------------------------------------

{\large 13 de Marzo}\\[2cm] % Date, change the \today to a set date if you want to be precise

%----------------------------------------------------------------------------------------
%	LOGO SECTION
%----------------------------------------------------------------------------------------

\includegraphics{logo.png}\\[1cm] % Include a department/university logo - this will require the graphicx package
 
%----------------------------------------------------------------------------------------

\vfill % Fill the rest of the page with whitespace

\end{titlepage}

%\todo[inline, color=green!40]{This is an inline comment.}
%\todo{Here's a comment in the margin!}
% Commands to include a figure:
%\begin{figure}
%\centering
%\includegraphics[width=0.5\textwidth]{frog.jpg}
%\caption{\label{fig:frog}This is a figure caption.}
%\end{figure}


\section{Análisis de la Eficiencia}


\subsection{Algoritmo de Inserción}
\begin{lstlisting}[language=C]
inline static void insercion(int T[], int num_elem)
{
  insercion_lims(T, 0, num_elem);
}

static void insercion_lims(int T[], int inicial, int final)
{
  int i, j;
  int aux;
  for (i = inicial + 1; i < final; i++) {
    j = i;
    while ((T[j] < T[j-1]) && (j > 0)) {
      aux = T[j];
      T[j] = T[j-1];
      T[j-1] = aux;
      j--;
    };
  };
}
\end{lstlisting}

Vamos a estudiar el peor caso que se le podría presentar al algoritmo de inserción, es decir, que el vector estuviera ordenado en orden inverso (de mayor a menor).

Como vemos en la línea 5, siempre se llama al método con los argumentos $0$ y $num\_elem$, por lo que a partir de ahora para nosotros, $inicial$ será $0$ y $final$ será $n$.

La mayor parte del tiempo de ejecución se emplea en el cuerpo del bucle $while$ interno. Ese trozo de código se puede acotar por una constante $a$. Por lo tanto, las líneas 15-19 se ejecutan un número de veces dependiente del bucle externo, exactamente $i$ veces (ya que estamos suponiendo que nos encontramos en el peor caso). El bucle externo se ejecuta exactamente $n$ veces, por lo que nos queda:

\[T(n)=\sum_{i=1}^{n-1}\sum_{j=1}^{i}a=a\sum_{i=0}^{n-1}\sum_{j=1}^{i}1=a\sum_{i=0}^{n-1}i=a\frac{n(n-1)}{2}\]
Por lo que vemos que $T(n)\in O(n^2)$ o cuadrático.

\subsection{Algoritmo QuickSort}
\begin{lstlisting}[language=C]

static void heapsort(int T[], int num_elem)
{
  int i;
  for (i = num_elem/2; i >= 0; i--)
    reajustar(T, num_elem, i);
  for (i = num_elem - 1; i >= 1; i--)
    {
      int aux = T[0];
      T[0] = T[i];
      T[i] = aux;
      reajustar(T, i, 0);
    }
} 

static void reajustar(int T[], int num_elem, int k)
{
  int j;
  int v;
  v = T[k];
  bool esAPO = false;
  while ((k < num_elem/2) && !esAPO)
    {
      j = k + k + 1;

      if ((j < (num_elem - 1)) && (T[j] < T[j+1])) j++;
      if (v >= T[j]) esAPO = true;
      
      T[k] = T[j];
      k = j;
    }
  T[k] = v;
}
\end{lstlisting}

Vemos que en el primer bucle de la función $heapsort$ aparece la función $reajustar$, por lo que vamos a calcular su eficiencia primero. 

Vemos que el cuerpo del bucle $while$ consume la mayor parte del tiempo de ejecución. El cuerpo de bucle se puede acotar por una constante $b$. Como vemos en la línea 22, el bucle empieza en $k$ y termina en $\frac{n}{2}$, pero $k$ no avanza de 1 en 1, sino de $2k+1$ en $2k+1$, por lo que como mucho se ejecutará $\log(n/2)$ veces. Por lo que $R(n) \in O(\log(n))$, siendo $K(n)$ la función de eficiencia de la función $reajustar$.
Una vez conocida la eficiencia de la función $reajustar$, pasamos a estudiar el primer bucle de la función $heapsort$. Va desde $n/2$ hasta $0$, así que se ejecuta $n/2$ veces, y en cada una de esas veces ejecuta la función $reajustar$, por lo que encontes su eficiencia es $\frac{nlog(n)}{2}$. El segundo bucle se ejecuta $n-1$ veces, por lo que nos queda:
\[T(n)=\frac{nlog(n)}{2}+(n-1)\]
Por lo que $T(n)\in O(n\log(n))$.

\subsection{Algoritmo de Floyd}
\begin{lstlisting}[language=C]
void Floyd(int **M, int dim)
{
	for (int k = 0; k < dim; k++)
	  for (int i = 0; i < dim;i++)
	    for (int j = 0; j < dim;j++)
	      {
				int sum = M[i][k] + M[k][j];    	
		    M[i][j] = (M[i][j] > sum) ? sum : M[i][j];
	      }
}
\end{lstlisting}

Vemos que el cuerpo del tercer bucle $while$ anidado consume la mayor parte del tiempo de ejecución, por lo que lo acotamos por una constante $a$. Fácilmente vemos que el resto de bucles va desde 0 hasta $dim$, que podemos denominar $n$ para mayor facilidad. Por lo que evidentemente tenemos que la eficiencia de este algoritmo es $an^3$, es decir, $T(n)\in O(n^3)$.

\end{document}
